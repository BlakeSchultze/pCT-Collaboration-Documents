\documentclass[class = article, landscape,crop = false]{standalone}
\RequirePackage[landscape]{my-latex}

\setlistdepth{12}

\newlist{myEnumerate}{enumerate}{12}
\setlist[myEnumerate,1]{label=\textbf{(\arabic*)},labelindent=1pt, leftmargin=* }
\setlist[myEnumerate,2]{label=\textbf{(\alph*)},labelindent=1pt, leftmargin=*}
\setlist[myEnumerate,3]{label=\textbf{(\roman*)},labelindent=1pt, leftmargin=*}
\setlist[myEnumerate,4]{label=\textbf{(\arabic*)},labelindent=1pt, leftmargin=*}
\setlist[myEnumerate,5]{label=\textbf{(\alph*)},labelindent=1pt, leftmargin=*}
\setlist[myEnumerate,6]{label=\textbf{(\roman*)},labelindent=1pt, leftmargin=*}
\setlist[myEnumerate,7]{label=\textbf{(\arabic*)},labelindent=1pt, leftmargin=*}
\setlist[myEnumerate,8]{label=\textbf{(\alph*)},labelindent=1pt, leftmargin=*}
\setlist[myEnumerate,9]{label=\textbf{(\roman*)},labelindent=1pt, leftmargin=*}
\setlist[myEnumerate,10]{label=\textbf{(\arabic*)},labelindent=1pt, leftmargin=*}
\setlist[myEnumerate,11]{label=\textbf{(\alph*)},labelindent=1pt, leftmargin=*}
\setlist[myEnumerate,12]{label=\textbf{(\roman*)},labelindent=1pt, leftmargin=*}


%\title{\vspace{-4ex}Formatting pCT Data for Storage on the Baylor Server: Folder/File Naming and Organizational Scheme}
%\title{\vspace{-4ex}Storing pCT Data on the Baylor Server: Folder/File Naming and Organizational Scheme}
%\title{\vspace{-4ex}pCT Data Storage Format for Baylor Server: Folder/File Naming and Organizational Scheme}
%\date{\vspace{-5ex}}

%%%%%%%%%%%%%%%%%%%%%%%%%%%%%%%%%%%%%%%%%%%%%%%%%%%%%%%%%%%%%%%%%%%%%%%%%%%%%%%%%%%%%%%%%%%%%%%%%%%%%%%%%%%%%%%%%%%%%%%%%%%%%%%%%%%%%%%%%%%%%%%%%%%%%%%%%%%%%%%%%%%%%%%%%%%%%%%%%%%%%%%%%%%%%%%%%%%%%%%%%%%%%%%%%%%%%%%%%%%%%%
%%%%%%%%%%%%%%%%%%%%%%%%%%%%%%%%%%%%%%%%%%%%%%%%%%%%%%%%%%%%%%%%%%%%%%%%%%%%%%%%%%%%%%%%%%%%%%%%%%%% Document Body %%%%%%%%%%%%%%%%%%%%%%%%%%%%%%%%%%%%%%%%%%%%%%%%%%%%%%%%%%%%%%%%%%%%%%%%%%%%%%%%%%%%%%%%%%%%%%%%%%%%%%%%%%%
%%%%%%%%%%%%%%%%%%%%%%%%%%%%%%%%%%%%%%%%%%%%%%%%%%%%%%%%%%%%%%%%%%%%%%%%%%%%%%%%%%%%%%%%%%%%%%%%%%%%%%%%%%%%%%%%%%%%%%%%%%%%%%%%%%%%%%%%%%%%%%%%%%%%%%%%%%%%%%%%%%%%%%%%%%%%%%%%%%%%%%%%%%%%%%%%%%%%%%%%%%%%%%%%%%%%%%%%%%%%%%
%\begin{document}
%\pagestyle{fancy}
%{\chead{\flushleft\vspace{-0.3cm}\includegraphics[width=0.12\textwidth]{Baylor_IM_horz.png}}}

%\cfoot{\flushleft\vspace{-0.5cm}\includegraphics[width=0.15\textwidth]{Baylor_IM_horz.png}}
\begin{documentation}
	%[]
    	%{Storing pCT Data on the Baylor Server}
    	{pCT Data Storage on Baylor Server}
    	{Folder/File Naming and Organizational Scheme}

%\flushleft{\textbf{Bold and Underlined}} : Folder names (1) whose names are constant in \color{DarkGreen}{green}\color{Black} and (2) whose names vary depending on object name or date in \color{Brown}{brown}.
%%%%%%%%%%%%%%%%%%%%%%%%%%%%%%%%%%%%%%%%%%%%%%%%%%%%%%%%%%%%%%%%%%%%%%%%%%%%%%%%%%%%%%%%%%%%%%%%%%%%%%%%%%%%%%%%%%%%%%%%%%%%%%%%%%%%%%%%%%%%%%%%%%%%%%%%%%%%%%%%%%%%%%%%%%%%%%%%%%%%%%%%%%%%%%%%%%%%%%%%%%%%%%%%%%%%%%%%%%%%%%
\vspace{-3mm}
\sectionheader{Description of User/Shared Directories on Kodiak/Tardis Cluster Nodes}
%%%%%%%%%%%%%%%%%%%%%%%%%%%%%%%%%%%%%%%%%%%%%%%%%%%%%%%%%%%%%%%%%%%%%%%%%%%%%%%%%%%%%%%%%%%%%%%%%%%%%%%%%%%%%%%%%%%%%%%%%%%%%%%%%%%%%%%%%%%%%%%%%%%%%%%%%%%%%%%%%%%%%%%%%%%%%%%%%%%%%%%%%%%%%%%%%%%%%%%%%%%%%%%%%%%%%%%%%%%%%%
\vspace{-3mm}
\textbf{\color{DarkGreen}{\\\dirsep home\dirsep\usernamelabel}}\color{Black} : This is the user's private home directory where the user account's settings/history files (.bash\_profile, .bash\_history, etc.) are stored. Each user only has access to their directory but they can access it from each of the Kodiak/Tardis nodes. These directories have a limited storage capacity and their contents are not backed up, meaning its contents will not be recoverable in the event of failure.  Hence, these should not be used to store user data/code, particularly if it is large and/or important.\\
%\vspace{-5mm}
%%%%%%%%%%%%%%%%%%%%%%%%%%%%%%%%%%%%%%%%%%%%%%%%%%%%%%%%%%%%%%%%%%%%%%%%%%%%%%%%%%%%%%%%%%%%%%%%%%%%%%%%%%%%%%%%%%%%%%%%%%%%%%%%%%%%%%%%%%%%%%%%%%%%%%%%%%%%%%%%%%%%%%%%%%%%%%%%%%%%%%%%%%%%%%%%%%%%%%%%%%%%%%%%%%%%%%%%%%%%%%
\textbf{\color{DarkGreen}{\\\dirsep data\dirsep\usernamelabel}}\color{Black} : This is the global home directory of each user where the input/output data for code/program execution and all permanent/important personal data should be stored.  As a subdirectory of \textbf{\color{DarkGreen}{\dirsep data}}\color{Black}, the contents of these directories are backed up periodically, but like the local home directories, each user only has access to their personal directory.\\
%\vspace{-5mm}
%%%%%%%%%%%%%%%%%%%%%%%%%%%%%%%%%%%%%%%%%%%%%%%%%%%%%%%%%%%%%%%%%%%%%%%%%%%%%%%%%%%%%%%%%%%%%%%%%%%%%%%%%%%%%%%%%%%%%%%%%%%%%%%%%%%%%%%%%%%%%%%%%%%%%%%%%%%%%%%%%%%%%%%%%%%%%%%%%%%%%%%%%%%%%%%%%%%%%%%%%%%%%%%%%%%%%%%%%%%%%%
\textbf{\color{DarkGreen}{\\\dirsep ion\dirsep$\dots$ }}\color{Black} : This directory is dedicated to proton and ion therapy research and this is where users can upload data/code and selectively share portions of it with the rest of the collaboration by moving the desired files to the staging area (\textbf{\color{DarkGreen}{\dirsep staging\dirsep\usernamelabel}}\color{Black})and an administrator will then move them to the appropriate shared data/code directory.  The data in this directory is located on a network storage device and this drive is mounted on both the Kodiak and Tardis clusters, making it accessible from all of the master/compute nodes.  To prevent a permanent loss of data in the event of drive failure, the contents of this directory are periodically backed up to tape drive.\\
%\vspace{-5mm}
%%%%%%%%%%%%%%%%%%%%%%%%%%%%%%%%%%%%%%%%%%%%%%%%%%%%%%%%%%%%%%%%%%%%%%%%%%%%%%%%%%%%%%%%%%%%%%%%%%%%%%%%%%%%%%%%%%%%%%%%%%%%%%%%%%%%%%%%%%%%%%%%%%%%%%%%%%%%%%%%%%%%%%%%%%%%%%%%%%%%%%%%%%%%%%%%%%%%%%%%%%%%%%%%%%%%%%%%%%%%%%
\textbf{\color{DarkGreen}{\\\dirsep ion\dirsep incoming\dirsep\usernamelabel}}\color{Black} : This is the directory where users should upload data/code to the Baylor server prior to preparing and organizing it for sharing with the collaboration, with each user only having access to their personal directory in \textbf{\color{DarkGreen}{\dirsep incoming}}\color{Black}.  When a user is ready to share data/code, it should be moved to their directory in the staging area and an administrator will move it to the appropriate shared directory.\\
%\vspace{-5mm}
%%%%%%%%%%%%%%%%%%%%%%%%%%%%%%%%%%%%%%%%%%%%%%%%%%%%%%%%%%%%%%%%%%%%%%%%%%%%%%%%%%%%%%%%%%%%%%%%%%%%%%%%%%%%%%%%%%%%%%%%%%%%%%%%%%%%%%%%%%%%%%%%%%%%%%%%%%%%%%%%%%%%%%%%%%%%%%%%%%%%%%%%%%%%%%%%%%%%%%%%%%%%%%%%%%%%%%%%%%%%%%
\textbf{\color{DarkGreen}{\\\dirsep ion\dirsep staging\dirsep\usernamelabel}}\color{Black} : This is the directory where users move data/code they wish to share with the collaboration, at which point an administrator will move the data from the staging area to the appropriate shared directory.  Since the administrators are not familiar with pCT or its organizational scheme, it is important for users to organize the data/code prior to adding it to the staging area so an administrator can easily determine and move the data to the appropriate shared directory. 


In particular, users should create the entire target hierarchy of shared subdirectories below \textbf{\color{DarkGreen}{\dirsep ion}}\color{Black} inside the staging area.  The administrator can then follow these hierarchies downward until encountering a subdirectory in the staging area hierarchy that does not currently exist in the target shared directory, at which point they can simply move this subdirectory and its contents from the staging area to the target shared directory.  The subdirectories of the target shared directory that already exist must still be created and included in the staging area hierarchy so the directory hierarchies in the staging area and target shared directory are identical below \textbf{\color{DarkGreen}{\dirsep ion} }\color{Black} and \textbf{\color{DarkGreen}{\\\dirsep ion\dirsep incoming\dirsep\usernamelabel}}\color{Black}, respectively, providing the administrator with the target data path and required organizational framework and prevents the need to create/name/organize subdirectories.\\
%%%%%%%%%%%%%%%%%%%%%%%%%%%%%%%%%%%%%%%%%%%%%%%%%%%%%%%%%%%%%%%%%%%%%%%%%%%%%%%%%%%%%%%%%%%%%%%%%%%%%%%%%%%%%%%%%%%%%%%%%%%%%%%%%%%%%%%%%%%%%%%%%%%%%%%%%%%%%%%%%%%%%%%%%%%%%%%%%%%%%%%%%%%%%%%%%%%%%%%%%%%%%%%%%%%%%%%%%%%%%%
\textbf{\color{DarkGreen}{\\\dirsep ion\dirsep pCT\_data\dirsep$\dots$ }}\color{Black} : This directory is where the raw, preprocessed, projection, and reconstruction data/images are moved to make them available to the other pCT users.  Each type of data is stored in separate subdirectories and soft links to this data are created and organized in a directory hierarchy indicating their input/output data dependencies.  The directory/file naming and organizational scheme for each type of data and the soft links are outlined in the next section of this document.  Data/images should only be moved to this shared directory after having been verified as valid/accurate and having been organized appropriately.\\
%\vspace{-5mm}
%%%%%%%%%%%%%%%%%%%%%%%%%%%%%%%%%%%%%%%%%%%%%%%%%%%%%%%%%%%%%%%%%%%%%%%%%%%%%%%%%%%%%%%%%%%%%%%%%%%%%%%%%%%%%%%%%%%%%%%%%%%%%%%%%%%%%%%%%%%%%%%%%%%%%%%%%%%%%%%%%%%%%%%%%%%%%%%%%%%%%%%%%%%%%%%%%%%%%%%%%%%%%%%%%%%%%%%%%%%%%%
\textbf{\color{DarkGreen}{\\\dirsep ion\dirsep pCT\_code\dirsep $\dots$ }}\color{Black} : This directory contains the source code for the programs involved in the pCT reconstruction process, from data acquisition to image reconstruction; this includes programs use to generate simulated pCT data, preprocess raw experimental data, reconstruct images from experimental/simulated data, and reconstructed image analysis tools.  This directory also contains the code/program (BAP-WED\_Analysis) which calculates the water equivalent depth (WED) for a user supplied list of targeting beam angles and beam aim point (BAP) coordinates using the reconstructed RSP values for the target volume.  

The various programs are stored in separate directories and each of these directories contains a ``\textit{master}'' directory containing the current release version of the program as well as individual directories for each developer's code branch containing experimental features currently in development.  Users contribute new features to the program by merging these into the ``\textit{master}'' branch so that the next release version includes them. Note that merges with the ``\textit{master}'' branch are not necessarily made available as a new release version immediately, several merges from multiple developers may be performed before the corresponding additions/modifications are made available in a new release version.  Each program and its branches are now available on GitHub as separate repositories under the ``pCT\_collaboration'' account, separated as such so the list of users having ``push'' access to each program's repository can be controlled individually.  Cloning each repository to Kodiak and linking it to GitHub provides a mechanism to check for and automatically apply available updates prior to using the code by entering \texttt{git pull --rebase}.\\
%\vspace{-5mm}
%%%%%%%%%%%%%%%%%%%%%%%%%%%%%%%%%%%%%%%%%%%%%%%%%%%%%%%%%%%%%%%%%%%%%%%%%%%%%%%%%%%%%%%%%%%%%%%%%%%%%%%%%%%%%%%%%%%%%%%%%%%%%%%%%%%%%%%%%%%%%%%%%%%%%%%%%%%%%%%%%%%%%%%%%%%%%%%%%%%%%%%%%%%%%%%%%%%%%%%%%%%%%%%%%%%%%%%%%%%%%%
\textbf{\color{DarkGreen}{\\\dirsep ion\dirsep pCT\_data\dirsep pCT\_Documentation\dirsep $\dots$ }}\color{Black} : Documentation relevant to pCT is stored in this directory, such as descriptions of the data format, coordinate system, and phantoms and pCT related publications (including student theses/dissertations).  This is a GitHub managed local repository allowing everyone to ``push'' contributions to the repository and ``pull'' updates/additions from others into their own local clone ensuring everyone has access to the latest information.\\
\baylorsection
%%%%%%%%%%%%%%%%%%%%%%%%%%%%%%%%%%%%%%%%%%%%%%%%%%%%%%%%%%%%%%%%%%%%%%%%%%%%%%%%%%%%%%%%%%%%%%%%%%%%%%%%%%%%%%%%%%%%%%%%%%%%%%%%%%%%%%%%%%%%%%%%%%%%%%%%%%%%%%%%%%%%%%%%%%%%%%%%%%%%%%%%%%%%%%%%%%%%%%%%%%%%%%%%%%%%%%%%%%%%%%
\vspace{-9mm}
\flushleft\shadowtext{\anttmcap14pt\ul{Key:}}\\
\vspace{-0.1cm}
\normalfont
\begin{enumerate}[label={}]
\item \color{DarkGreen}{\textbf{Green}}\color{Black}\textbf{ : directories whose names do not change}\\
\item \color{Brown}{\textbf{Brown}}\color{Black}\textbf{ : directories whose names vary depending on object name or date}\\
\item \color{Black}\textbf{\color{DodgerBlue3}{\emph{Italic/Royal Blue}}\color{Black}}\textbf{ : individual files}\\
\item \color{Black}\textbf{\color{DarkBlue}{\emph{Italic/Dark Blue}}\color{Black}}\textbf{ : multiple files }\\
\end{enumerate}
%\rule{\textwidth}{1pt}
\vspace{-6mm}
\baylorsection
%\vspace{0.8cm}
%%%%%%%%%%%%%%%%%%%%%%%%%%%%%%%%%%%%%%%%%%%%%%%%%%%%%%%%%%%%%%%%%%%%%%%%%%%%%%%%%%%%%%%%%%%%%%%%%%%%%%%%%%%%%%%%%%%%%%%%%%%%%%%%%%%%%%%%%%%%%%%%%%%%%%%%%%%%%%%%%%%%%%%%%%%%%%%%%%%%%%%%%%%%%%%%%%%%%%%%%%%%%%%%%%%%%%%%%%%%%%
%\vspace{-9mm}
\clearpage
\textbf{\color{DarkGreen}{\dirsep data\dirsep ion\dirsep pCT\_data\dirsep organized\_data} \color{Black}: directory containing all of the pCT data (raw, processed, projection, and reconstruction), primarily as soft links to the actual data itself (with each type of data stored/organized in separate directories), organized in a hierarchy of subdirectories indicating data dependencies.}
    \begin{myEnumerate}[labelindent=1pt, leftmargin=*]
        \item \color{Brown}{{\textbf{\dirsep Phantom}}} \color{Black}: directory containing all of the experimental/simulated data and reconstructed images associated with the phantom/object named ``\textit{Phantom}''.
        \begin{myEnumerate}[labelindent=1pt, leftmargin=*]
            \item \color{DarkGreen}{{\textbf{\dirsep Reference\_Images}}} \color{Black}: directory containing reference images (xCT, RSP, etc) relevant to analysis/comparison of the data/images for this object and data type.
            \item \color{DarkGreen}{{\textbf{\dirsep Experimental}}} \color{Black}: directory containing data and images generated from an experimental scan of the object.
            \begin{myEnumerate}[labelindent=1pt, leftmargin=*]
                \item \color{Brown}{{\textbf{\dirsep YY-MM-DD}}} \color{Black}: directory containing data and reconstructed images corresponding to all experimental scans of the object performed on this date.
                \begin{myEnumerate}[labelindent=1pt, leftmargin=*]
                    \item \color{Brown}{{\textbf{\dirsep XXXX[\_AAA]}}} \color{Black}: directory containing data and reconstructed images corresponding to the experimental scan of the object for this particular run \# of this date, where the run \# is of the form $\textrm{``\textit{XXXX}''}$ with an optional descriptor tag $\textrm{``\textit{\_AAA}''}$ added specifying additional pertinent information about the scan, such as inferior $\textrm{``\textit{\_Inf}''}$ or superior $\textrm{``\textit{\_Sup}''}$ positioning as in the case of a head phantom.
                    \begin{myEnumerate}[labelindent=1pt, leftmargin=*]
                        \item \color{DarkGreen}{{\textbf{\dirsep Input}}} \color{Black}: directory containing raw data generated by scan of object from each gantry angle and transmitted by event builder.
                        \begin{myEnumerate}[labelindent=1pt, leftmargin=*]
                            \item \color{DarkBlue}{\textbf{\textit{raw\_xxx.bin}}} \color{Black}: binary files containing trigger/tracker/energy detector data from event builder associated with gantry position $\textrm{``\textit{xxx}''} =\{\textrm{``\textit{001}''}, \textrm{``\textit{002}''}, \textrm{``\textit{003}''}, \cdots\}$.
                        \end{myEnumerate}
                        \item \color{DarkGreen}{{\textbf{\dirsep Output}}} \color{Black}: directory containing calibration and post processed data generated from analysis of raw data and used as input to image reconstruction.
                        \begin{myEnumerate}[labelindent=1pt, leftmargin=*]
                            \item \color{Brown}{{\textbf{\dirsep YY-MM-DD}}} \color{Black}: directory containing the post processed ``\textit{projection\_xxx.bin}'' data generated on this date and the reconstructions using this data.
                            \begin{myEnumerate}[labelindent=1pt, leftmargin=*]
                                \item \color{DodgerBlue3}{\textbf{\textit{calib.txt}}} \color{Black}: text file containing calibration curve coefficients for WEPL calibration.
                                \item \color{DodgerBlue3}{\textbf{\textit{TVcalib.txt}}} \color{Black}: text file containing calibration curve coefficients for tv corrected WEPL calibration.
                                \item \color{DodgerBlue3}{\textbf{\textit{settings.cfg}}} \color{Black}: specifies scan properties such as gantry angle interval, t/v detector size, reconstruction volume dimensions, etc and initial settings to use for preprocessing/reconstruction.
                                \item \color{DarkBlue}{\textbf{\textit{projection\_xxx.bin}}} \color{Black}: binary files containing tracker coordinates and WEPL data associated with gantry position $\textrm{``\textit{xxx}''} =\{\textrm{``\textit{001}''}, \textrm{``\textit{002}''}, \textrm{``\textit{003}''}, \cdots\}$ converted from raw data using tracker alignment, track reconstruction, and WEPL calibration routines and used as input to image reconstruction.
                                \item \color{DarkGreen}{{\textbf{\dirsep Reconstruction}}} \color{Black}: directory containing preprocessed data and reconstructed images generated using the ``\textit{projection\_xxx.bin}'' data along with reference images relevant to the object.
                                \begin{myEnumerate}[labelindent=1pt, leftmargin=*]
                                    \item \color{Brown}{{\textbf{\dirsep YY-MM-DD}}} \color{Black}: directory containing the preprocessed data generated on this date and the reconstructed images generated from this data.
                                    \begin{myEnumerate}[labelindent=1pt, leftmargin=*]
                                        \item \color{DodgerBlue3}{\textbf{\textit{settings\_log.cfg}}} \color{Black}: copy of \textbf{\textit{settings.cfg}} with any changes made to parameters/options applied at execution, if any.
                                        \item \color{DodgerBlue3}{\textbf{\textit{execution\_times.txt}}} \color{Black}: execution times for various portions of preprocessing and/or reconstruction and total program execution time.
                                        \item \color{DodgerBlue3}{\textbf{\textit{bin\_counts.txt}}} \color{Black}: linearized bin \# for each proton history, where linearized bin \# = t\_bin + angle\_bin * T\_BINS + v\_bin * T\_BINS * ANGULAR\_BINS.\\
                                        \item {\color{DodgerBlue3}{\textbf{\textit{mean\_rel\_ut\_angle.txt}}} \color{Black}: mean relative ut angle ($\angle ut_{out} -\angle ut_{in}$) by linearized bin \#.}\\
                                        \item \color{DodgerBlue3}{\textbf{\textit{mean\_rel\_uv\_angle.txt}}} \color{Black}: mean relative uv angle ($\angle uv_{out} -\angle uv_{in}$) by linearized bin \#.
                                        \item \color{DodgerBlue3}{\textbf{\textit{mean\_WEPL.txt}}} \color{Black}: mean WEPL value by linearized bin \#.
                                        \item \color{DodgerBlue3}{\textbf{\textit{stddev\_rel\_ut\_angle.txt}}} \color{Black}: standard deviation of the relative ut angle ($\angle ut_{out} -\angle ut_{in}$) by linearized bin \#.
                                        \item \color{DodgerBlue3}{\textbf{\textit{stddev\_rel\_uv\_angle.txt}}} \color{Black}: standard deviation of the relative uv angle ($\angle uv_{out} -\angle uv_{in}$) by linearized bin \#.
                                        \item \color{DodgerBlue3}{\textbf{\textit{stddev\_WEPL.txt}}} \color{Black}: standard deviation of the WEPL value by linearized bin \#.
                                        \item \color{DodgerBlue3}{\textbf{\textit{sinogram.txt}}} \color{Black}: mean WEPL after statistical cuts with the $t_{bin}$ and angular bin $\theta_{bin}$ plane for each vertical bin $v_{bin}$ stacked on each other.
                                        \item \color{DodgerBlue3}{\textbf{\textit{hull.txt}}} \color{Black}: text file specifying hull in 1s/0s with the xy plane for each slice stacked on each other.
                                        \item \color{DodgerBlue3}{\textbf{\textit{FBP.txt}}} \color{Black}: text file specifying filtered back projection image with the xy plane for each slice stacked on each other.
                                        \item \color{DodgerBlue3}{\textbf{\textit{x\_0.txt}}} \color{Black}: text file specifying voxel values of initial iterate with the xy plane for each slice stacked on each other.
                                        \item \color{DodgerBlue3}{\textbf{\textit{sin\_table.bin}}} \color{Black}: file containing the tabulated values of sine function
                                        \item \color{DodgerBlue3}{\textbf{\textit{cos\_table.bin}}} \color{Black}: file containing the tabulated values of cosine function
                                        \item \color{DodgerBlue3}{\textbf{\textit{coefficient.bin}}} \color{Black}: file containing the tabulated scattering coefficient values for $\Sigma_1$/$\Sigma_2$ for $u_2-u_1$/$u_1$ values
                                        \item \color{DodgerBlue3}{\textbf{\textit{poly\_1\_2.bin}}} \color{Black}: file containing the tabulated MLP polynomial values with coefficients $\{1,2,3,4,5,6\}$
                                        \item \color{DodgerBlue3}{\textbf{\textit{poly\_2\_3.bin}}} \color{Black}: file containing the tabulated MLP polynomial values with coefficients $\{2,3,4,5,6,7\}$
                                        \item \color{DodgerBlue3}{\textbf{\textit{poly\_3\_4.bin}}} \color{Black}: file containing the tabulated MLP polynomial values with coefficients $\{3,4,5,6,7,8\}$
                                        \item \color{DodgerBlue3}{\textbf{\textit{poly\_2\_6.bin}}} \color{Black}: file containing the tabulated MLP polynomial values with coefficients $\{2,6,12,20,30,42\}$
                                        \item \color{DodgerBlue3}{\textbf{\textit{poly\_3\_12.bin}}} \color{Black}: file containing the tabulated MLP polynomial values with coefficients $\{3,12,30,60,105,168\}$
                                        \item \color{DodgerBlue3}{\textbf{\textit{MLP.bin}}} \color{Black}: binary file with MLP path data for each history entering hull.
                                        \item \color{DodgerBlue3}{\textbf{\textit{WEPL.bin}}} \color{Black}: binary file specifying WEPL value for each history entering hull.
                                        \item \color{DodgerBlue3}{\textbf{\textit{histories.bin}}} \color{Black}: binary file specifying entry/exit coordinates/angles, bin number, gantry angle, and hull entry x/y/z voxel \# for each history entering hull.
                                        \item \color{DarkGreen}{{\textbf{\dirsep Images}}} \color{Black}: directory containing reconstructed images generated using this preprocessed data.
                                        \begin{myEnumerate}[labelindent=1pt, leftmargin=*]
                                            \item \color{Brown}{{\textbf{\dirsep YY-MM-DD}}} \color{Black}: directory containing the reconstructed images generated on this date using the preprocessed data above.
                                                \begin{myEnumerate}[labelindent=1pt, leftmargin=*]
                                                    \item \color{DarkBlue}{\textbf{\textit{x\_k.dcm}}} \color{Black}: DICOM images of x after $k$ iterations.
                                                    \item \color{DarkBlue}{\textbf{\textit{x\_k.txt}}} \color{Black}: text images of x after $k$ iterations.
                                                    \item \color{DarkBlue}{\textbf{\textit{x\_k.png}}} \color{Black}: PNG images of x after $k$ iterations.
                                                \end{myEnumerate}
                                        \end{myEnumerate}
                                    \end{myEnumerate}
                                \end{myEnumerate}
                            \end{myEnumerate}
                        \end{myEnumerate}
                    \end{myEnumerate}
                \end{myEnumerate}
            \end{myEnumerate}
            %\newpage
            \item \color{DarkGreen}{{\textbf{/Simulated}}} \color{Black}: directory containing data and images generated from a simulated scan of the object.
            \begin{myEnumerate}[labelindent=1pt, leftmargin=*]
                \item \color{Brown}{{\textbf{\dirsep G\_YY-MM-DD}}} \color{Black}: directory containing data and reconstructed images corresponding to all GEANT4 simulated scans of the object generated on this date.
                \begin{myEnumerate}[labelindent=1pt, leftmargin=*]
                    \item \color{Brown}{{\textbf{\dirsep XXXX[\_AAA]}}} \color{Black}: directory containing data and reconstructed images corresponding to the experimental scan of the object for this particular run \# of this date, where the run \# is of the form $\textrm{``\textit{XXXX}''}$ with an optional descriptor tag $\textrm{``\textit{\_AAA}''}$ added specifying additional pertinent information about the scan, such as inferior $\textrm{``\textit{\_Inf}''}$ or superior $\textrm{``\textit{\_Sup}''}$ positioning as in the case of a head phantom.
                    \begin{myEnumerate}[labelindent=1pt, leftmargin=*]
                        \item \color{DarkGreen}{{\textbf{\dirsep Input}}} \color{Black}: directory containing raw data files generated by simulated scan of object for each gantry angle.
                        \begin{myEnumerate}[labelindent=1pt, leftmargin=*]
                            \item \color{DarkBlue}{\textbf{\textit{raw\_xxx.bin}}} \color{Black}: binary files containing trigger/tracker/energy detector data from event builder associated with gantry position $\textrm{``\textit{xxx}''} =\{\textrm{``\textit{001}''}, \textrm{``\textit{002}''}, \textrm{``\textit{003}''}, \cdots\}$.
                        \end{myEnumerate}
                        \item \color{DarkGreen}{{\textbf{\dirsep Output}}} \color{Black}: directory containing calibration and post processed data generated from analysis of raw data and used as input to image reconstruction.
                        \begin{myEnumerate}[labelindent=1pt, leftmargin=*]
                            \item \color{Brown}{{\textbf{\dirsep YY-MM-DD}}} \color{Black}: directory containing the post processed ``\textit{projection\_xxx.bin}'' data generated on this date and the reconstructions using this data.
                            \begin{myEnumerate}[labelindent=1pt, leftmargin=*]
                                \item \color{DodgerBlue3}{\textbf{\textit{calib.txt}}} \color{Black}: text file containing calibration curve coefficients for WEPL calibration.
                                \item \color{DodgerBlue3}{\textbf{\textit{TVcalib.txt}}} \color{Black}: text file containing calibration curve coefficients for tv corrected WEPL calibration.
                                \item \color{DarkBlue}{\textbf{\textit{projection\_xxx.bin}}} \color{Black}: binary files containing tracker coordinates and WEPL data associated with gantry position $\textrm{``\textit{xxx}''} =\{\textrm{``\textit{001}''}, \textrm{``\textit{002}''}, \textrm{``\textit{003}''}, \cdots\}$ converted from raw data using WEPL calibration routine and used as input to image reconstruction.
                                \item \color{DarkGreen}{{\textbf{\dirsep Reconstruction}}} \color{Black}: directory containing preprocessed data and reconstructed images generated using the ``\textit{projection\_xxx.bin}'' data along with reference images relevant to the object.
                                \begin{myEnumerate}[labelindent=1pt, leftmargin=*]
                                    \item \color{Brown}{{\textbf{\dirsep YY-MM-DD}}} \color{Black}: directory containing the preprocessed data generated on this date and the reconstructed images generated from this data.
                                    \begin{myEnumerate}[labelindent=1pt, leftmargin=*]
                                        \item \color{DodgerBlue3}{\textbf{\textit{hull.txt}}} \color{Black}: text file specifying hull in 1s/0s.
                                        \item \color{DodgerBlue3}{\textbf{\textit{FBP.txt}}} \color{Black}: text file specifying filtered back projection image.
                                        \item \color{DodgerBlue3}{\textbf{\textit{x\_0.txt}}} \color{Black}: text file specifying voxel values of initial iterate.
                                        \item \color{DodgerBlue3}{\textbf{\textit{MLP.bin}}} \color{Black}: binary file with MLP path data for each history entering hull.
                                        \item \color{DodgerBlue3}{\textbf{\textit{WEPL.bin}}} \color{Black}: binary file specifying WEPL value for each history entering hull.
                                        \item \color{DodgerBlue3}{\textbf{\textit{histories.bin}}} \color{Black}: binary file specifying entry/exit coordinates/angles, bin number, gantry angle, and hull entry x/y/z voxel \# for each history entering hull.
                                        \item \color{DarkGreen}{{\textbf{\dirsep Images}}} \color{Black}: directory containing reconstructed images generated using this preprocessed data.
                                        \begin{myEnumerate}[labelindent=1pt, leftmargin=*]
                                            \item \color{Brown}{{\textbf{\dirsep YY-MM-DD}}} \color{Black}: directory containing the reconstructed images generated on this date using the preprocessed data above.
                                                \begin{myEnumerate}[labelindent=1pt, leftmargin=*]
                                                    \item \color{DarkBlue}{\textbf{\textit{x\_k.dcm}}} \color{Black}: DICOM images of x after $k$ iterations.
                                                    \item \color{DarkBlue}{\textbf{\textit{x\_k.txt}}} \color{Black}: text images of x after $k$ iterations.
                                                    \item \color{DarkBlue}{\textbf{\textit{x\_k.png}}} \color{Black}: PNG images of x after $k$ iterations.
                                                \end{myEnumerate}
                                        \end{myEnumerate}
                                    \end{myEnumerate}
                                \end{myEnumerate}
                            \end{myEnumerate}
                        \end{myEnumerate}
                    \end{myEnumerate}
                \end{myEnumerate}
                \item \color{Brown}{{\textbf{\dirsep T\_YY-MM-DD}}} \color{Black}: directory containing data and reconstructed images corresponding to all TOPAS simulated scans of the object generated on this date.
                \begin{myEnumerate}[labelindent=1pt, leftmargin=*]
                    \item \color{Brown}{{\textbf{\dirsep XXXX[\_AAA]}}} \color{Black}: directory containing data and reconstructed images corresponding to the experimental scan of the object for this particular run \# of this date, where the run \# is of the form $\textrm{``\textit{XXXX}''}$ with an optional descriptor tag $\textrm{``\textit{\_AAA}''}$ added specifying additional pertinent information about the scan, such as inferior $\textrm{``\textit{\_Inf}''}$ or superior $\textrm{``\textit{\_Sup}''}$ positioning as in the case of a head phantom.
                    \begin{myEnumerate}[labelindent=1pt, leftmargin=*]
                        \item \color{DarkGreen}{{\textbf{\dirsep Input}}} \color{Black}: directory containing raw data files generated by simulated scan of object for each gantry angle.
                        \begin{myEnumerate}[labelindent=1pt, leftmargin=*]
                            \item \color{DarkBlue}{\textbf{\textit{raw\_xxx.bin}}} \color{Black}: binary files containing trigger/tracker/energy detector data from event builder associated with gantry position $\textrm{``\textit{xxx}''} =\{\textrm{``\textit{001}''}, \textrm{``\textit{002}''}, \textrm{``\textit{003}''}, \cdots\}$.
                        \end{myEnumerate}
                        \item \color{DarkGreen}{{\textbf{\dirsep Output}}} \color{Black}: directory containing calibration and post processed data generated from analysis of raw data and used as input to image reconstruction.
                        \begin{myEnumerate}[labelindent=1pt, leftmargin=*]
                            \item \color{Brown}{{\textbf{\dirsep YY-MM-DD}}} \color{Black}: directory containing the post processed ``\textit{projection\_xxx.bin}'' data generated on this date and the reconstructions using this data.
                            \begin{myEnumerate}[labelindent=1pt, leftmargin=*]
                                \item \color{DodgerBlue3}{\textbf{\textit{calib.txt}}} \color{Black}: text file containing calibration curve coefficients for WEPL calibration.
                                \item \color{DodgerBlue3}{\textbf{\textit{TVcalib.txt}}} \color{Black}: text file containing calibration curve coefficients for tv corrected WEPL calibration.
                                \item \color{DarkBlue}{\textbf{\textit{projection\_xxx.bin}}} \color{Black}: binary files containing tracker coordinates and WEPL data associated with gantry position $\textrm{``\textit{xxx}''} =\{\textrm{``\textit{001}''}, \textrm{``\textit{002}''}, \textrm{``\textit{003}''}, \cdots\}$ converted from raw data using WEPL calibration routine and used as input to image reconstruction.
                                \item \color{DarkGreen}{{\textbf{\dirsep Reconstruction}}} \color{Black}: directory containing preprocessed data and reconstructed images generated using the ``\textit{projection\_xxx.bin}'' data along with reference images relevant to the object.
                                \begin{myEnumerate}[labelindent=1pt, leftmargin=*]
                                    \item \color{Brown}{{\textbf{\dirsep YY-MM-DD}}} \color{Black}: directory containing the preprocessed data generated on this date and the reconstructed images generated from this data.
                                    \begin{myEnumerate}[labelindent=1pt, leftmargin=*]
                                        \item \color{DodgerBlue3}{\textbf{\textit{hull.txt}}} \color{Black}: text file specifying hull in 1s/0s.
                                        \item \color{DodgerBlue3}{\textbf{\textit{FBP.txt}}} \color{Black}: text file specifying filtered back projection image.
                                        \item \color{DodgerBlue3}{\textbf{\textit{x\_0.txt}}} \color{Black}: text file specifying voxel values of initial iterate.
                                        \item \color{DodgerBlue3}{\textbf{\textit{MLP.bin}}} \color{Black}: binary file with MLP path data for each history entering hull.
                                        \item \color{DodgerBlue3}{\textbf{\textit{WEPL.bin}}} \color{Black}: binary file specifying WEPL value for each history entering hull.
                                        \item \color{DodgerBlue3}{\textbf{\textit{histories.bin}}} \color{Black}: binary file specifying entry/exit coordinates/angles, bin number, gantry angle, and hull entry x/y/z voxel \# for each history entering hull.
                                        \item \color{DarkGreen}{{\textbf{\dirsep Images}}} \color{Black}: directory containing reconstructed images generated using this preprocessed data.
                                        \begin{myEnumerate}[labelindent=1pt, leftmargin=*]
                                            \item \color{Brown}{{\textbf{\dirsep YY-MM-DD}}} \color{Black}: directory containing the reconstructed images generated on this date using the preprocessed data above.
                                                \begin{myEnumerate}[labelindent=1pt, leftmargin=*]
                                                    \item \color{DarkBlue}{\textbf{\textit{x\_k.dcm}}} \color{Black}: DICOM images of x after $k$ iterations.
                                                    \item \color{DarkBlue}{\textbf{\textit{x\_k.txt}}} \color{Black}: text images of x after $k$ iterations.
                                                    \item \color{DarkBlue}{\textbf{\textit{x\_k.png}}} \color{Black}: PNG images of x after $k$ iterations.
                                                \end{myEnumerate}
                                        \end{myEnumerate}
                                    \end{myEnumerate}
                                \end{myEnumerate}
                            \end{myEnumerate}
                        \end{myEnumerate}
                    \end{myEnumerate}
                \end{myEnumerate}
            \end{myEnumerate}
        \end{myEnumerate}
    \end{myEnumerate}
\flushleft\vspace{-15mm}
\baylorsection
%\clearpage
%%%%%%%%%%%%%%%%%%%%%%%%%%%%%%%%%%%%%%%%%%%%%%%%%%%%%%%%%%%%%%%%%%%%%%%%%%%%%%%%%%%%%%%%%%%%%%%%%%%%%%%%%%%%%%%%%%%%%%%%%%%%%%%%%%%%%%%%%%%%%%%%%%%%%%%%%%%%%%%%%%%%%%%%%%%%%%%%%%%%%%%%%%%%%%%%%%%%%%%%%%%%%%%%%%%%%%%%%%%%%
\vspace{-9mm}
%\leavevmode\unskip\ignorespaces\begin{flushleft}
%\leavevmode\color{DarkGreen}\textbf{\dirsep data\dirsep ion\dirsep pCT\_data/raw\_data}\textbf{: directory containing the raw experimental data organized by scan date}
%\end{flushleft}
\flushleft\color{DarkGreen}{\textbf{\dirsep data\dirsep ion\dirsep pCT\_data/raw\_data}} \color{Black}\textbf{: directory containing the raw experimental data organized by scan date}
\begin{myEnumerate}[labelindent=1pt, leftmargin=*]
    \item \color{Brown}{{\textbf{\dirsep YY-MM-DD}}} \color{Black}: Folder containing all raw experimental data acquired from the scan beginning on ``\textit{YY-MM-DD}''
    \begin{myEnumerate}[labelindent=1pt, leftmargin=*]
        \item \color{DarkBlue}{\textbf{\textit{$<$object$>$\_XXXX[\_AAA]\_xxx.dat}}} \color{Black}: raw experimental data for the object named ``\textit{$<$object$>$}'', from run \# ``\textit{XXXX[\_AAA]}'', where ``\textit{XXXX}'' is a 4 digit \# with leading zeros and ``\textit{\_AAA}'' is an optional descriptor tag, and ``\textit{xxx}'' is the gantry angle at which the data was acquired.
    \end{myEnumerate}
\end{myEnumerate}
\vspace{-7mm}
\baylorsection
%%%%%%%%%%%%%%%%%%%%%%%%%%%%%%%%%%%%%%%%%%%%%%%%%%%%%%%%%%%%%%%%%%%%%%%%%%%%%%%%%%%%%%%%%%%%%%%%%%%%%%%%%%%%%%%%%%%%%%%%%%%%%%%%%%%%%%%%%%%%%%%%%%%%%%%%%%%%%%%%%%%%%%%%%%%%%%%%%%%%%%%%%%%%%%%%%%%%%%%%%%%%%%%%%%%%%%%%%%%%%
%\rule{\textwidth}{1pt}
\vspace{-9mm}
\flushleft\color{DarkGreen}{\textbf{\dirsep data\dirsep ion\dirsep pCT\_data/preprocessed\_data}} \color{Black}\textbf{: directory containing the preprocessed experimental data organized by scan and processed dates}
\begin{myEnumerate}[labelindent=1pt, leftmargin=*]
    \item \color{Brown}{{\textbf{\dirsep YY-MM-DD}}} \color{Black}: Folder containing all processed experimental data corresponding to the raw experimental data acquired on ``\textit{YY-MM-DD}''
    \begin{myEnumerate}[labelindent=1pt, leftmargin=*]
        \item \color{Brown}{{\textbf{\dirsep YY-MM-DD}}} \color{Black}: Folder containing all processed experimental data generated on ``\textit{YY-MM-DD}'' from the raw data
        \begin{myEnumerate}[labelindent=1pt, leftmargin=*]
            \item \color{DarkBlue}{\textbf{\textit{$<$object$>$\_XXXX[\_AAA]\_xxx.dat.root.reco.root.bin}}} \color{Black}: processed experimental data with tracker coordinates, recovery of missing hits when possible, and calibrated WEPL measurements for the object named ``\textit{$<$object$>$}'', from run \# ``\textit{XXXX[\_AAA]}'', where ``\textit{XXXX}'' is a 4 digit \# with leading zeros and ``\textit{\_AAA}'' is an optional descriptor tag, and ``\textit{xxx}'' is the gantry angle at which the data was acquired.
        \end{myEnumerate}
    \end{myEnumerate}
\end{myEnumerate}
%%%%%%%%%%%%%%%%%%%%%%%%%%%%%%%%%%%%%%%%%%%%%%%%%%%%%%%%%%%%%%%%%%%%%%%%%%%%%%%%%%%%%%%%%%%%%*******%%%%%%%%%%%%%%%%%%%%%%%%%%%%%%%%%%%%%%%%%%%%%%%%%%%%%%%%%%%%%%%%%%%%%%%%%%%%%%%%%%%%%%%%%%%%%%%%%%%%%%%%%%%%%%%%%%%%%%%%%

\sectionheader{File List}
\vspace{-9mm}
\flushleft
 \color{DarkBlue}{\textbf{\textit{$<$object$>$\_XXXX[\_AAA]\_xxx.dat}}} \color{Black}: raw experimental data for the object named ``\textit{$<$object$>$}'', from run \# ``\textit{XXXX[\_AAA]}'', where ``\textit{XXXX}'' is a 4 digit \# with leading zeros and ``\textit{\_AAA}'' is an optional descriptor tag, and ``\textit{xxx}'' is the gantry angle at which the data was acquired.
 \\
\color{DarkBlue}{\textbf{\textit{$<$object$>$\_XXXX[\_AAA]\_xxx.dat.root.reco.root.bin}}} \color{Black}: processed experimental data with tracker coordinates, recovery of missing hits when possible, and calibrated WEPL measurements for the object named ``\textit{$<$object$>$}'', from run \# ``\textit{XXXX[\_AAA]}'', where ``\textit{XXXX}'' is a 4 digit \# with leading zeros and ``\textit{\_AAA}'' is an optional descriptor tag, and ``\textit{xxx}'' is the gantry angle at which the data was acquired.
\\
\color{DodgerBlue3}{\textbf{\textit{settings\_log.cfg}}} \color{Black}: copy of \textbf{\textit{settings.cfg}} with any changes made to parameters/options applied at execution, if any.
\\
\color{DodgerBlue3}{\textbf{\textit{execution\_times.txt}}} \color{Black}: Listing of the preprocessing/reconstruction options/parameters used for a particular reconstruction along with the execution times for various individual tasks and portions of preprocessing and/or reconstruction and total program execution time organized as a series of ``key=value'' entries in a text file in the directory with the corresponding results.
\\
\color{DodgerBlue3}{\textbf{\textit{execution\_times.csv}}} \color{Black}: The same information listed in the \color{DodgerBlue3}{\textbf{\textit{execution\_times.txt}}} \color{Black} files for each execution are also stored in this global comma separated value (.csv) file with the execution information organized into separate columns and each row corresponding to a different reconstruction, thereby maintaining a single file where the execution information for all previous reconstructions can be found in the same place.  Upon execution of a new reconstruction, a new row entry is added for storage of the corresponding execution information.
\\
\color{DodgerBlue3}{\textbf{\textit{bin\_counts.txt}}} \color{Black}: linearized bin \# for each proton history, where linearized bin \# = t\_bin + angle\_bin * T\_BINS + v\_bin * T\_BINS * ANGULAR\_BINS.
\\
{\color{DodgerBlue3}{\textbf{\textit{mean\_rel\_ut\_angle.txt}}} \color{Black}: mean relative ut angle ($\angle ut_{out} -\angle ut_{in}$) by linearized bin \#.}
\\
\color{DodgerBlue3}{\textbf{\textit{mean\_rel\_uv\_angle.txt}}} \color{Black}: mean relative uv angle ($\angle uv_{out} -\angle uv_{in}$) by linearized bin \#.
\\
\color{DodgerBlue3}{\textbf{\textit{mean\_WEPL.txt}}} \color{Black}: mean WEPL value by linearized bin \#.
\\
\color{DodgerBlue3}{\textbf{\textit{stddev\_rel\_ut\_angle.txt}}} \color{Black}: standard deviation of the relative ut angle ($\angle ut_{out} -\angle ut_{in}$) by linearized bin \#.
\\
\color{DodgerBlue3}{\textbf{\textit{stddev\_rel\_uv\_angle.txt}}} \color{Black}: standard deviation of the relative uv angle ($\angle uv_{out} -\angle uv_{in}$) by linearized bin \#.
\\
\color{DodgerBlue3}{\textbf{\textit{stddev\_WEPL.txt}}} \color{Black}: standard deviation of the WEPL value by linearized bin \#.
\\
\color{DodgerBlue3}{\textbf{\textit{sinogram.txt}}} \color{Black}: mean WEPL after statistical cuts with the $t_{bin}$ and angular bin $\theta_{bin}$ plane for each vertical bin $v_{bin}$ stacked on each other.
\\
\color{DodgerBlue3}{\textbf{\textit{hull.txt}}} \color{Black}: text file specifying hull in 1s/0s with the xy plane for each slice stacked on each other.
\\
\color{DodgerBlue3}{\textbf{\textit{FBP.txt}}} \color{Black}: text file specifying filtered back projection image with the xy plane for each slice stacked on each other.
\\
\color{DodgerBlue3}{\textbf{\textit{x\_0.txt}}} \color{Black}: text file specifying voxel values of initial iterate with the xy plane for each slice stacked on each other.
\\
\color{DodgerBlue3}{\textbf{\textit{sin\_table.bin}}} \color{Black}: file containing the tabulated values of sine function
\\
\color{DodgerBlue3}{\textbf{\textit{cos\_table.bin}}} \color{Black}: file containing the tabulated values of cosine function
\\
\color{DodgerBlue3}{\textbf{\textit{coefficient.bin}}} \color{Black}: file containing the tabulated scattering coefficient values for $\Sigma_1$/$\Sigma_2$ for $u_2-u_1$/$u_1$ values
\\
\color{DodgerBlue3}{\textbf{\textit{poly\_1\_2.bin}}} \color{Black}: file containing the tabulated MLP polynomial values with coefficients $\{1,2,3,4,5,6\}$
\\
\color{DodgerBlue3}{\textbf{\textit{poly\_2\_3.bin}}} \color{Black}: file containing the tabulated MLP polynomial values with coefficients $\{2,3,4,5,6,7\}$
\\
\color{DodgerBlue3}{\textbf{\textit{poly\_3\_4.bin}}} \color{Black}: file containing the tabulated MLP polynomial values with coefficients $\{3,4,5,6,7,8\}$
\\
\color{DodgerBlue3}{\textbf{\textit{poly\_2\_6.bin}}} \color{Black}: file containing the tabulated MLP polynomial values with coefficients $\{2,6,12,20,30,42\}$
\\
\color{DodgerBlue3}{\textbf{\textit{poly\_3\_12.bin}}} \color{Black}: file containing the tabulated MLP polynomial values with coefficients $\{3,12,30,60,105,168\}$
\\
\color{DodgerBlue3}{\textbf{\textit{MLP.bin}}} \color{Black}: binary file with MLP path data for each history entering hull.
\\
\color{DodgerBlue3}{\textbf{\textit{WEPL.bin}}} \color{Black}: binary file specifying WEPL value for each history entering hull.
\\
\color{DodgerBlue3}{\textbf{\textit{histories.bin}}} \color{Black}: binary file specifying entry/exit coordinates/angles, bin number, gantry angle, and hull entry x/y/z voxel \# for each history entering hull.
\\
\color{DodgerBlue3}{\textbf{\textit{calib.txt}}} \color{Black}: text file containing calibration curve coefficients for WEPL calibration.
\\
\color{DodgerBlue3}{\textbf{\textit{TVcalib.txt}}} \color{Black}: text file containing calibration curve coefficients for tv corrected WEPL calibration.
\\
\color{DarkBlue}{\textbf{\textit{projection\_xxx.bin}}} \color{Black}: binary files containing tracker coordinates and WEPL data associated with gantry position $\textrm{``\textit{xxx}''} =\{\textrm{``\textit{001}''}, \textrm{``\textit{002}''}, \textrm{``\textit{003}''}, \cdots\}$ converted from raw data using WEPL calibration routine and used as input to image reconstruction.
\\
\color{DarkBlue}{\textbf{\textit{raw\_xxx.bin}}} \color{Black}: binary files containing trigger/tracker/energy detector data from event builder associated with gantry position $\textrm{``\textit{xxx}''} =\{\textrm{``\textit{001}''}, \textrm{``\textit{002}''}, \textrm{``\textit{003}''}, \cdots\}$.
\\
bin\_counts.txt
\\
bin\_counts\_post.txt
\\
bin\_counts\_pre.txt
\\
execution\_times.csv
\\
execution\_times.txt
\\
FBP\_hull.txt
\\
FBP\_median\_filtered.txt
\\
hull\_avg\_filtered.txt
\\
MSC\_counts.txt
\\
MSC\_hull.txt
\\
sinogram\_pre.txt
\\
\color{DarkBlue}{\textbf{\textit{x\_k.dcm}}} \color{Black}: DICOM images of x after $k$ iterations.
\color{DarkBlue}{\textbf{\textit{x\_k.txt}}} \color{Black}: text images of x after $k$ iterations.
\color{DarkBlue}{\textbf{\textit{x\_k.png}}} \color{Black}: PNG images of x after $k$ iterations.
\vspace{-7mm}
\baylorsection
%%%%%%%%%%%%%%%%%%%%%%%%%%%%%%%%%%%%%%%%%%%%%%%%%%%%%%%%%%%%%%%%%%%%%%%%%%%%%%%%%%%%%%%%%%%%%*******%%%%%%%%%%%%%%%%%%%%%%%%%%%%%%%%%%%%%%%%%%%%%%%%%%%%%%%%%%%%%%%%%%%%%%%%%%%%%%%%%%%%%%%%%%%%%%%%%%%%%%%%%%%%%%%%%%%%%%%%%
\vspace{-9mm}

\vspace{-7mm}
\baylorsection
%%%%%%%%%%%%%%%%%%%%%%%%%%%%%%%%%%%%%%%%%%%%%%%%%%%%%%%%%%%%%%%%%%%%%%%%%%%%%%%%%%%%%%%%%%%%%*******%%%%%%%%%%%%%%%%%%%%%%%%%%%%%%%%%%%%%%%%%%%%%%%%%%%%%%%%%%%%%%%%%%%%%%%%%%%%%%%%%%%%%%%%%%%%%%%%%%%%%%%%%%%%%%%%%%%%%%%%%
\end{documentation}
%\end{document}
\color{Brown}{{\textbf{\dirsep YY-MM-DD}}} \color{Black}: Folder containing all raw experimental data acquired from the scan beginning on ``\textit{YY-MM-DD}''


Administrators can then follow the path downward until either the data's target directory is reached or
For example, data from a new run date should be organized in the staging area in a hierarchy of directories including all parent directories ( object name, simulated/experimental scan type, run date) below \textbf{\color{DarkGreen}{\dirsep ion}}\color{Black} such that the hierarchy of directories below \textbf{\color{DarkGreen}{\dirsep ion\dirsep staging\dirsep\usernamelabel\dirsep$\dots$}}\color{Black} mirrors the path to its target location.  The administrator can then follow the target path downward from \textbf{\color{DarkGreen}{\dirsep ion}}\color{Black} creating any directories that do not already exist and moving the data/code into their corresponding target directories.  For example, when data from a new run date is moved to the staging area, it should be organized in the hierarchy of directories corresponding to the object, scan type, run date, run \#, etc.  If data already exists for this object, this folder will already exist and need not be created.  If it is the first data set for a particular object, then the administrator can simply move the entire object folder and its contents to the target path and the

downward through directories that already exist along the target path and create those that do not and copy the data and any folders

data/code in the staging directory so the administrator can .

and other PDF documents.   so they need only locate the corresponding directory in the shared directories and will simply move files from the staging area to the appropriate shared directory the files in the staging area must be organized properly so an administrator can determine which shared directory it should be moved to.  Administrators typically will not be familiar with pCT or the organizational scheme

It is important to organize the data/code added to the staging area prior to contacting an administrator to move it to a shared directory Prior to contacting an administrator to move When a user is ready to share data with the other pCT users, it should be moved to this directory and a member of the ``\textit{ionadm}'' group will move this to the appropriate shared directory.  This data must be organized according to the format outlined in this document for data of this type. 

Data/code that a user wants to share with the collaboration should first be organized into a hierarchy of subdirectories according to the established organizational scheme and then moved to their personal directory in the staging area (\textbf{\color{DarkGreen}{\dirsep ion\dirsep staging\dirsep\usernamelabel}}\color{Black}).  Administrators are not familiar with pCT or its organizational scheme but if the relevant subdirectories are created/organized appropriately, they need only move the data and subdirectories from the staging area to the corresponding shared directories as well as the subdirectories and their contents


The organization step is particularly important since administrators are not familiar with pCT data/code or its organizational scheme and will simply , making the organization step particularly important as an administrator will use then move the files to a shared directory accessible by the entire collaboration.  In general, users should consider removing data from their incoming directory once it has been added to the staging area and moved to a shared directory to avoid unnecessary duplicate data and memory consumption.
\vspace{-5mm}
%%%%%%%%%%%%%%%%%%%%%%%%%%%%%%%%%%%%%%%%%%%%%%%%%%%%%%%%%%%%%%%%%%%%%%%%%%%%%%%%%%%%%%%%%%%%%%%%%%%%%%%%%%%%%%%%%%%%%%%%%%%%%%%%%%%%%%%%%%%%%%%%%%%%%%%%%%%%%%%%%%%%%%%%%%%%%%%%%%%%%%%%%%%%%%%%%%%%%%%%%%%%%%%%%%%%%%%%%%%%%%
\flushleft\textbf{\color{DarkGreen}{\\\dirsep ion\dirsep staging\dirsep\usernamelabel }}\color{Black} : This is the directory where users should move data/code when it is ready to be shared with the collaboration and contact.  At the moment, users should then notify a cluster administrator (e.g. Brian Sitton, Keith Schubert) when there is data/code ready to be moved to a shared directory, but work will be done to make this a more streamlined process.  Note that administrators typically will not be familiar with pCT or the organizational scheme so it important for the files in the staging area to be organized such that their path below \textbf{\color{DarkGreen}{\dirsep ion\dirsep staging\dirsep\usernamelabel }}\color{Black} exactly mirrors the path below \textbf{\color{DarkGreen}{\dirsep ion}}\color{Black} where it is to be moved.

In other words, the data in the staging directory should be organized into the exact same hierarchy of directories as it is to be organized in the target subdirectory of \textbf{\color{DarkGreen}{\dirsep ion}}\color{Black}, as defined by the pCT data/code organizational scheme, with the top directory in the staging area corresponding to the first subdirectory of \textbf{\color{DarkGreen}{\dirsep ion }}\color{Black} along the target path (often \textbf{\color{DarkGreen}{\dirsep pCT\_data }}\color{Black} or \textbf{\color{DarkGreen}{\dirsep pCT\_code}}\color{Black}).  The administrator can then move the data/code to the appropriate target destination by following the target and staging area paths downward from \textbf{\color{DarkGreen}{\dirsep ion }}\color{Black} and \textbf{\color{DarkGreen}{staging\dirsep\usernamelabel }}\color{Black}, respectively, moving data from the staging area to the corresponding target directory as it is encountered.  This is repeated until all files have been moved or a subdirectory along the target path does not exist in the target hierarchy, at which point the missing subdirectory and its contents can then be moved directly from the staging area to the target directory to complete the move. 