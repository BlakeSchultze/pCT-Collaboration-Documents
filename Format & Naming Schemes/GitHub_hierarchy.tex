%\clearpage%
\Section{GitHub Accounts/Repositories}
\begin{tcbfunctionenv}'tcbBlueStyle'[title=\shadowText{GitHub}]
%%%%%%%%%%%%%%%%%%%%%%%%%%%%%%%%%%%%%%%%%%%%%%%%%%%%%%%%%%%%%%%%%%%%%%%%%%%%%%%%%%%%%%%%%%%%%%%%%%%%%%%%%%%%%%%%%%%%%%%%%%%%%%%%%%%%%%%%%%%%%%%%%%%%%%%%%%%%%%%%%%%%%%%%%%%%%%%%%%%%%%%%%%%%%%%%%%%%%%%%%%%%%%%%%%%%%%%%%%%%%%
\begin{tcbparagraph}'tcbBlueStyle'(title=https:\dirsep\dirsep github.com\dirsep$<$\emph{GitHub account}$>$\dirsep$<$\emph{GitHub repository}$>$)
$\boldsymbol{-}$ below is an outline of the GitHub accounts and repositories containing documentation, tools, and code relevant to the storage of pCT data/code, data acquisition, processing, and analysis, and the various programs and tools used to perform pCT related tasks
\end{tcbparagraph}
%%%%%%%%%%%%%%%%%%%%%%%%%%%%%%%%%%%%%%%%%%%%%%%%%%%%%%%%%%%%%%%%%%%%%%%%%%%%%%%%%%%%%%%%%%%%%%%%%%%%%%%%%%%%%%%%%%%%%%%%%%%%%%%%%%%%%%%%%%%%%%%%%%%%%%%%%%%%%%%%%%%%%%%%%%%%%%%%%%%%%%%%%%%%%%%%%%%%%%%%%%%%%%%%%%%%%%%%%%%%%%
\begin{tcbparagraph}'tcbBlueStyle'{tcbenumeratedStyle}
\begin{deepList}[labelindent=1pt, leftmargin=*]
	\item \coloredtext*{documentation-gitaccount}{\dirsep pCT-collaboration} : parent directory for all pCT code/data on Kodiak and the Tardis compute nodes
	\begin{deepList}[labelindent=1pt, leftmargin=*]
		\item \coloredtext*{documentation-gitrepo}{\dirsep pCT\_Tools} : contains bash functions/scripts and other tools useful for navigating data/code and configuring/running programs on Kodiak and Tardis compute nodes along with documentation describing them and their purpose/usage.  There is also a default \coloredtext*{documentation-file}{\texttt{.bash\_profile}} for pCT users which sources \coloredtext*{documentation-file}{\texttt{pct\_user\_script.sh}} to configure user sessions depending on the current Kodiak/Tardis node and \coloredtext*{documentation-file}{\texttt{load\_pct\_functions.sh}} to automatically load the aforementioned bash functions useful during a user terminal session.
		\item \coloredtext*{documentation-gitrepo}{\dirsep pCT-docs} : contains documentation on the pCT data/code naming and organizational scheme, collaborator's project involvement and contact info, and phantom naming (including relevant subcategory tags) and properties/manuals.  The contents of this repository are primarily being added from the original \coloredtext*{documentation-gitrepo}{\dirsep BlakeSchultze\dirsep pCT\_Documentation} documentation repository, but the naming and contents of these files are being verified/improved prior to adding them to this repository and some of the original files may not be added
This repository is currently incomplete
		\item \coloredtext*{documentation-gitrepo}{\dirsep pypct} : Python helpers for proton CT
		\item \coloredtext*{documentation-gitrepo}{\dirsep pct-acquire} : parent directory for all pCT code/data on Kodiak and the Tardis compute nodes
		\item \coloredtext*{documentation-gitrepo}{\dirsep pct-sim} : Geant4 simulations for proton CT
		\item \coloredtext*{documentation-gitrepo}{\dirsep Preprocessing} : program for preprocessing raw data to generate tracker coordinates and perform WEPL calibration.
		\item \coloredtext*{documentation-gitrepo}{\dirsep pct-recon-copy} : original pCT reconstruction program developed by Penfold/Hurley, now including separate branches for execution on Tardis compute nodes using modified Makefile and addition of execution parameters for input/output directory allowing these to be specified at runtime without requiring recompilation, a capability required for development of a batch script to submit reconstruction job(s) to the GPU execution queue.
		\item \coloredtext*{documentation-gitrepo}{\dirsep Reconstruction\_BU} : contains only the current and previous release versions of Baylor's reconstruction program as developed in \coloredtext*{documentation-gitaccount}{\dirsep BaylorICTHUS\dirsep pCT\_Reconstruction} (no code development is performed here).
	\end{deepList}
%%%%%%%%%%%%%%%%%%%%%%%%%%%%%%%%%%%%%%%%%%%%%%%%%%%%%%%%%%%%%%%%%%%%%%%%%%%%%%%%%%%%%%%%%%%%%%%%%%%%%%%%%%%%%%%%%%%%%%%%%%%%%%%%%%%%%%%%%%%%%%%%%%%%%%%%%%%%%%%%%%%%%%%%%%%%%%%%%%%%%%%%%%%%%%%%%%%%%%%%%%%%%%%%%%%%%%%%%%%%%%
\newpage
%%%%%%%%%%%%%%%%%%%%%%%%%%%%%%%%%%%%%%%%%%%%%%%%%%%%%%%%%%%%%%%%%%%%%%%%%%%%%%%%%%%%%%%%%%%%%%%%%%%%%%%%%%%%%%%%%%%%%%%%%%%%%%%%%%%%%%%%%%%%%%%%%%%%%%%%%%%%%%%%%%%%%%%%%%%%%%%%%%%%%%%%%%%%%%%%%%%%%%%%%%%%%%%%%%%%%%%%%%%%%%
%%%%%%%%%%%%%%%%%%%%%%%%%%%%%%%%%%%%%%%%%%%%%%%%%%%%%%%%%%%%%%%%%%%%%%%%%%%%%%%%%%%%%%%%%%%%%%%%%%%%%%%%%%%%%%%%%%%%%%%%%%%%%%%%%%%%%%%%%%%%%%%%%%%%%%%%%%%%%%%%%%%%%%%%%%%%%%%%%%%%%%%%%%%%%%%%%%%%%%%%%%%%%%%%%%%%%%%%%%%%%%
	\item \coloredtext*{documentation-gitaccount}{\dirsep BaylorICTHUS} : Baylor's pCT programs, tools, and documentation.
	\begin{deepList}[labelindent=1pt, leftmargin=*]
		\item \coloredtext*{documentation-gitrepo}{\dirsep pCT\_Reconstruction} : used in developing the release version of Baylor's pCT reconstruction program and containing branches for each of Baylor's pCT developers (Blake, Paniz, Sarah, ...) for independent development relevant to their work.  Developments made in a developer branch and proposed for integration in the next release version go through a review and testing process to verify the code and its impact on the full program.  Developments passing this verification process are then merged into the \coloredtext*{documentation-gitbranch}{release\_development} branch.  When critical developments are merged into the \coloredtext*{documentation-gitbranch}{release\_development} branch, this branch is then merged into the \coloredtext*{documentation-gitbranch}{release} branch and the resulting code is then pushed to the \coloredtext*{documentation-gitrepo}{\dirsep pCT-collaboration\dirsep Reconstruction\_BU} repository, as this is the source for pCT users to acquire the current and previous release versions of the program.
	\end{deepList}
%%%%%%%%%%%%%%%%%%%%%%%%%%%%%%%%%%%%%%%%%%%%%%%%%%%%%%%%%%%%%%%%%%%%%%%%%%%%%%%%%%%%%%%%%%%%%%%%%%%%%%%%%%%%%%%%%%%%%%%%%%%%%%%%%%%%%%%%%%%%%%%%%%%%%%%%%%%%%%%%%%%%%%%%%%%%%%%%%%%%%%%%%%%%%%%%%%%%%%%%%%%%%%%%%%%%%%%%%%%%%%
	\item \coloredtext*{documentation-gitaccount}{\dirsep BlakeSchultze} : parent directory for all pCT code/data on Kodiak and the Tardis compute nodes
	\begin{deepList}[labelindent=1pt, leftmargin=*]
		\item \coloredtext*{documentation-gitrepo}{\dirsep LaTeX-Packages} : provides the package \coloredtext*{documentation-file}{``my-latex.sty''} which is included in TeX documents to provide access to the definitions of new commands/macros/environments, load the external/3rd-party package dependencies, and configure the typesetting of LaTeX documents as well as providing the collection of LaTeX style (.sty) and other files included in this repository upon which these definitions/configurations are dependent.
		\item \coloredtext*{documentation-gitrepo}{\dirsep pCT\_Documentation} : contains an expanded set of pCT documentation files with additional resources not included in the \coloredtext*{documentation-gitrepo}{\dirsep pCT-collaboration\dirsep pCT-docs} repository, such as pCT publications and theses/dissertations.
		\item \coloredtext*{documentation-gitrepo}{\dirsep pCT\_Reconstruction} : the original repository in which Baylor's pCT reconstruction program was developed, which also contains the experimental development of an alternative program configuration with several automated routines, and is currently being merged into the release version of Baylor's reconstruction program as provided in \coloredtext*{documentation-gitrepo}{\dirsep BaylorICTHUS\dirsep pCT\_Reconstruction}.
		\item \coloredtext*{documentation-gitrepo}{\dirsep WED\_Analysis} : provides tool for determining the water-equivalent depth (WED) for a set of beam-aim point (BAP) coordinates based on reconstructed image RSP values, using the voxel walk algorithm developed as part of the pCT reconstruction program.  This algorithm steps from voxel edge to voxel edge along a trajectory to determine exact voxel intersection coordinates and prevent the missing of small voxel intersections which can occur when taking constant length steps along a path as was done in the original reconstruction program.
	\end{deepList}
\end{deepList}
\end{tcbparagraph}
%\par\bigskip\bigskip\bigskip\bigskip\bigskip\smallskip
\end{tcbfunctionenv}

%\clearpage 